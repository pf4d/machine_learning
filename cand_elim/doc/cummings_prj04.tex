\documentclass[12pt]{article}

\usepackage{algorithm} %pseudo-code
\usepackage{algpseudocode}
\usepackage[top=1in, bottom=1in, left=1in, right=1in]{geometry}
\usepackage{amsmath,amsthm,amssymb}
\usepackage{graphicx}
\usepackage{float}
\usepackage{amsmath, amssymb, amscd}
\usepackage{alltt}
\usepackage{textcomp}
\usepackage{gensymb}
\usepackage{multicol}
\usepackage{tabularx}

\newcommand{\N}{\mathcal{N}}
\newcommand{\Z}{\mathbb{Z}}
\newcommand{\R}{\mathbb{R}}
\newcommand{\bigo}{\mathcal{O}}
\newcommand{\G}{\mathcal{G}}
\newcommand{\V}{\mathcal{V}}
\newcommand{\E}{\mathcal{E}}
\newcommand{\K}{\mathcal{K}}
\newcommand{\T}{^\intercal}

\newcommand{\sups}[1]{\ensuremath{^{\textrm{#1}}}}
\newcommand{\subs}[1]{\ensuremath{_{\textrm{#1}}}}

\newcommand{\specialcell}[2][c]
{
  \begin{tabular}[#1]{@{}c@{}}#2\end{tabular}
}

\makeatletter
\newsavebox{\mybox}\newsavebox{\mysim}
\newcommand{\distras}[1]
{
  \savebox{\mybox}{\hbox{\kern3pt$\scriptstyle#1$\kern3pt}}%
  \savebox{\mysim}{\hbox{$\sim$}}%
  \mathbin{\overset{#1}{\kern\z@\resizebox{\wd\mybox}{\ht\mysim}{$\sim$}}}%
}
\makeatother
\makeatletter
\renewcommand*\env@matrix[1][c]{\hskip -\arraycolsep
  \let\@ifnextchar\new@ifnextchar
  \array{*\c@MaxMatrixCols #1}}
\makeatother
%===============================================================================
% code highlighting :
\usepackage{listings}

% define custom colors :
\usepackage{color}
\definecolor{bg}{rgb}{0.96,0.96,0.85}
\definecolor{deepblue}{rgb}{0,0,0.5}
\definecolor{deepred}{rgb}{0.6,0,0}
\definecolor{deepgreen}{rgb}{0,0.5,0}

\usepackage{xcolor}
\renewcommand{\lstlistlistingname}{Code Listings}
\renewcommand{\lstlistingname}{Code Listing}
\definecolor{gray}{gray}{0.5}
\colorlet{commentcolour}{green!50!black}

\colorlet{stringcolour}{red!60!black}
\colorlet{keywordcolour}{magenta!90!black}
\colorlet{exceptioncolour}{yellow!50!red}
\colorlet{commandcolour}{blue!60!black}
\colorlet{numpycolour}{blue!60!green}
\colorlet{literatecolour}{magenta!90!black}
\colorlet{promptcolour}{green!50!black}
\colorlet{specmethodcolour}{violet}
\colorlet{indendifiercolour}{green!70!white}

\newcommand{\framemargin}{5ex}

\newcommand{\literatecolour}{\textcolor{literatecolour}}

\newcommand\pythonstyle{\lstset{
%keepspaces=true,
language=python,
showtabs=true,
tab=,
tabsize=2,
basicstyle=\ttfamily\scriptsize,%\setstretch{.5},
stringstyle=\color{stringcolour},
showstringspaces=false,
alsoletter={1234567890},
otherkeywords={\ , \}, \{, \%, \&, \|},
keywordstyle=\color{keywordcolour}\bfseries,
emph={and,break,class,continue,def,yield,del,elif ,else,%
except,exec,finally,for,from,global,if,import,in,%
lambda,not,or,pass,print,raise,return,try,while,assert},
emphstyle=\color{blue}\bfseries,
emph={[2]True, False, None},
emphstyle=[2]\color{keywordcolour},
emph={[3]object,type,isinstance,copy,deepcopy,zip,enumerate,reversed,list,len,dict,tuple,xrange,append,execfile,real,imag,reduce,str,repr},
emphstyle=[3]\color{commandcolour},
emph={Exception,NameError,IndexError,SyntaxError,TypeError,ValueError,OverflowError,ZeroDivisionError},
emphstyle=\color{exceptioncolour}\bfseries,
%upquote=true,
morestring=[s]{"""}{"""},
morestring=[s]{'''}{'''},
commentstyle=\color{commentcolour}\slshape,
%emph={[4]1, 2, 3, 4, 5, 6, 7, 8, 9, 0},
emph={[4]ode, fsolve, sqrt, exp, sin, cos, arccos, pi,  array, norm, solve, dot, arange, , isscalar, max, sum, flatten, shape, reshape, find, any, all, abs, linspace, legend, quad, polyval,polyfit, hstack, concatenate,vstack,column_stack,empty,zeros,ones,rand,vander,grid,pcolor,eig,eigs,eigvals,svd,qr,tan,det,logspace,roll,min,mean,cumsum,cumprod,diff,vectorize,lstsq,cla,eye,xlabel,ylabel,squeeze,plot,median,std,hist},
emphstyle=[4]\color{numpycolour},
emph={[5]__init__,__add__,__mul__,__div__,__sub__,__call__,__getitem__,__setitem__,__eq__,__ne__,__nonzero__,__rmul__,__radd__,__repr__,__str__,__get__,__truediv__,__pow__,__name__,__future__,__all__},
emphstyle=[5]\color{specmethodcolour},
emph={[6]assert,range,yield},
emphstyle=[6]\color{keywordcolour}\bfseries,
% emph={[7]self},
% emphstyle=[7]\bfseries,
literate=*%
{:}{{\literatecolour:}}{1}%
{=}{{\literatecolour=}}{1}%
{-}{{\literatecolour-}}{1}%
{+}{{\literatecolour+}}{1}%
{*}{{\literatecolour*}}{1}%
{/}{{\literatecolour/}}{1}%
{!}{{\literatecolour!}}{1}%
%{(}{{\literatecolour(}}{1}%
%{)}{{\literatecolour)}}{1}%
{[}{{\literatecolour[}}{1}%
{]}{{\literatecolour]}}{1}%
{<}{{\literatecolour<}}{1}%
{>}{{\literatecolour>}}{1}%
{>>>}{{\textcolor{promptcolour}{>>>}}}{1}%
,%
breaklines=true,
breakatwhitespace= true,
%xleftmargin=\framemargin,
%xrightmargin=\framemargin,
aboveskip=1ex,
frame=trbl,
%frameround=tttt,
rulecolor=\color{black!40},
%framexleftmargin=\framemargin,
%framextopmargin=.1ex,
%framexbottommargin=.1ex,
%framexrightmargin=\framemargin,
%framexleftmargin=1mm, framextopmargin=1mm, frame=shadowbox, rulesepcolor=\color{blue},#1
%frame=tb,
backgroundcolor=\color{yellow!10}
}}

% Python environment
\lstnewenvironment{python}[1][]
{
  \pythonstyle
  \lstset{#1}
}
{}

% Python for external files
\newcommand\pythonexternal[1]
{{
  \pythonstyle
  \lstinputlisting{#1}
}}

% Python for inline
\newcommand\pythoninline[1]{{\pythonstyle\lstinline!#1!}}

% end code highlighting
%===============================================================================

\DeclareMathOperator*{\argmax}{arg\,max}

\usepackage[top=.5in, bottom=.5in, left=.5in, right=.5in]{geometry}
\usepackage{natbib}

\begin{document}
\small

\title{Project 04 - Candidate Elimination}
\author{Evan Cummings\\
CSCI 544 - Machine Learning}

\maketitle

\section*{Abstract}

The candidate elimination algorithm classifies a set of data with two possible classifications by creating two boundaries for a data-set, one being the most specific and the other the most general.  These boundaries uniquely describe the version space of the data and allow classification to be performed.  In order to create these boundaries, the algorithm iterates through the data and alters the specific and general boundaries such that they become minimally more general and specific, respectively.  The approach described here imposes the additional requirement that the positive values differ from the negative values in at least one dimension; this has the effect of reducing the complexity of the algorithm to the columns of the data instead of the rows.  We explain how the boundaries and thus version space may be derived using only the unique values for each dimension of the data-set.

\section*{Description}

let $\mathcal{P}_j$ and $\mathcal{N}_j$ be the sets of all possible and values for the dimension $j$ of the data-set $\mathcal{D}$ corresponding to positive and negative values, respectively.  With this, I claim that we can describe the specific boundary $\mathcal{S}$ and general boundary $\mathcal{G}$.

Before these boundaries are described, note that in order to create a set of boundaries that are both more specific than the most general or more general than the most specific, we require at least one positive entry; if all we have are negative instances, we have no idea what the specific dimension values might be.  The algorithm described here makes the further requirement that $\mathcal{P} \not\equiv \mathcal{N}$.

In order to describe the boundaries, let the values of $\mathcal{S}$ be either true or false, where the value is false if and only if position $i$ of $\mathcal{S}$ may be any possible value of dimension $i$, i.e., it is ``wild''.  Likewise, let the values of $\mathcal{G}$ be either true or false, where the value is true if and only if the most general boundary elements may have a specific value in position $i$.

Using this description, the intermediate values of the version space $\mathcal{V}$ may be derived from $\mathcal{S}$ by taking all combinations of $\mathcal{S}$ choosing from two elements up to $n-1$ where $n$ is the number of true positions in $\mathcal{S}$.  The remaining elements of $\mathcal{V}$ correspond to the most general boundary of $\mathcal{D}$ and are all possible one non-wild combinations of the set $\mathcal{G}$.

\section*{Determining $\mathcal{S}$ and $\mathcal{G}$}

In order to determine the values of $\mathcal{S}$ and $\mathcal{G}$, note that index $j$ of both sets will be true if none of $\mathcal{P}_j$ intersect with $\mathcal{N}_j$; we have definitive values for the specific boundary and thus general boundary as well.  

Alternatively, if the number of values in $\mathcal{N}_j$ is greater than the number of values in $\mathcal{P}_j$, we know that $\mathcal{S}_j$ is true and $\mathcal{G}_j$ is false; we have more possible values in dimension $j$ than what we have in $\mathcal{S}_j$, and hence have a specific but not general value for dimension $j$.  

Likewise, if instead the number of values in $\mathcal{N}_j$ is less than the number of values in $\mathcal{P}_j$, we know that both $\mathcal{S}_j$ and $\mathcal{G}_j$ is false; the positive values may be anything and thus so may the specific and general boundaries.  

Finally, if $\mathcal{P}_j$ is equivalent to $\mathcal{N}_j$, we know that $\mathcal{G}_j$ is false; the general boundary does not have a specific value for dimension $j$.  However, a subtle distinction for the value of $\mathcal{S}_j$ in this case exists.  If there is only one possible value for both $\mathcal{P}_j$ and $\mathcal{N}_j$, $\mathcal{S}_j$ is true, otherwise, it is false -- we have a specific value for the specific boundary in the case that all possible values of dimension $j$ are the same.

These four cases cover all possibilities for the different values of $\mathcal{P}$ and $\mathcal{N}$ and thus describe completely the specific and general boundaries, $\mathcal{S}$ and $\mathcal{G}$, of the version space corresponding to $\mathcal{D}$.  Pseudo-code is presented below, as well as a Python implementation demonstrating 10-fold cross-validated 99.8 - 100 \% accuracy for the data-set presented in the book.

\section*{Pseudo-code}

\begin{Algorithm}[H]{12cm}
  \caption{ - Modified Candidate Elimination}
  \begin{algorithmic} 
    \State \textbf{INPUTS}: 
    \State \ \ \ $\mathcal{D}$ - $m \times n$ Data matrix,
    \State \ \ \ $\mathcal{C}$ - Class array.
    \State \textbf{OUTPUT}: 
    \State \ \ \ $\mathcal{S}$ - Specific boundary descriptor,
    \State \ \ \ $\mathcal{G}$ - General boundary descriptor, 
    \State \ \ \ $\mathcal{V}$ - Version space.
    \\
    \hrulefill
    \Function{mCe}{$\mathcal{D},\mathcal{C}$}
      \State $\mathcal{S} := \emptyset$
      \State $\mathcal{G} := \emptyset$
      \State $\mathcal{V} := \emptyset$
      \For{$j \in [0,n]$}
        \State $\mathcal{P}_j :=$ unique positive values of $\mathcal{D}_{\cdot,j}$
        \State $\mathcal{N}_j :=$ unique positive values of $\mathcal{D}_{\cdot,j}$
        \If{$\#(\mathcal{P}_j \cap \mathcal{N}_{j}) = 0$}
          \State $\mathcal{S}_j := 1,\ \mathcal{G}_j := 1$
        \ElsIf{$\#(\mathcal{N}_j) > \#(\mathcal{P}_j)$}
          \State $\mathcal{S}_j := 1,\ \mathcal{G}_j := 0$
        \ElsIf{$\#(\mathcal{N}_j) < \#(\mathcal{P}_j)$}
          \State $\mathcal{S}_j := 0,\ \mathcal{G}_j := 0$
        \ElsIf{$\mathcal{P}_j \equiv \mathcal{N}_j$}
          \If{$\#(\mathcal{N}_j) = 1$ \textbf{and} $\#(\mathcal{P}_j) = 1$}
            \State $\mathcal{S}_j := 1$
          \Else
            \State $\mathcal{S}_j := 0$
          \EndIf
          \State $\mathcal{N}_j := 0$
        \EndIf
      \EndFor
      \State $\mathcal{T}_s := \{i : \mathcal{S}_i = 1\}$
      \State $\mathcal{T}_g := \{i : \mathcal{G}_i = 1\}$
      \State $\mathcal{S}_s := \{\binom{\mathcal{T}_s}{k} : k \in [2,l],\ l = \sum_i \mathcal{S}_i\}$
      \State $\mathcal{S}_g := \binom{\mathcal{T}_g}{1}$
      \State $\mathcal{V} := S_s \cup S_g$
    \EndFunction
  \end{algorithmic}
\end{Algorithm}


\section*{Python}

\pythonexternal{../src/cand_elim.py}


\end{document}




