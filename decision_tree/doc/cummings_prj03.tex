\input{functions.tex}
\DeclareMathOperator*{\argmax}{arg\,max}

\usepackage[top=.5in, bottom=.5in, left=.5in, right=.5in]{geometry}
\usepackage{natbib}

\begin{document}
\small

\title{Project 03 - Decision Tree}
\author{Evan Cummings\\
CSCI 544 - Machine Learning}

\maketitle

\section*{Abstract}

This is a test of a decision tree algorithm developed in Python and utilizing the Pylab packages.  To run this script simply type

\begin{center} \texttt{python decision\_tree.py} \end{center}

\noindent in the \texttt{src} directory.

The approach used to solve this programming problem was taken directly from \cite{mitchell}. 

\begin{figure}[H]
  \centering
		\includegraphics[width=0.8\textwidth]{images/DT.png}
  \caption{\scriptsize Dataset drawn from the \emph{Breast Cancer Wisconsin} Data set found in the UCI Machine Learning Repository.}
\end{figure}

\section*{Results}

\begin{figure}[H]
  \centering
		\includegraphics[width=1.0\textwidth]{images/DT_results.png}
  \caption{\scriptsize Results for the D.T.\ classification, 90.0\% accurate.}
\end{figure}

\section*{Rapid miner}

\begin{figure}[H]
  \centering
    \includegraphics[width=0.8\textwidth]{images/RM_main_process.png}
  \caption{\scriptsize Main process window for software Rapid Miner on the same data set.}
\end{figure}

\begin{figure}[H]
  \centering
    \includegraphics[width=0.5\textwidth]{images/RM_tree.png}
  \caption{\scriptsize Part of the tree output from the Rapid Miner configuration above.  A bug in the program prevented the full output from being saved.}
\end{figure}


\section*{Code}

\pythonexternal{../src/decision_tree.py}

\bibliographystyle{alpha}
\bibliography{biblio}

\end{document}




